%%
%%   N  K  Y  M
%%    A  A  A  A
%%

%% typeset: latexmk (uplatex -> dvipdfmx)
\documentclass[uplatex,dvipdfmx]{vkaishi}

% \usepackage{v-hyperref}
\usepackage{vuccaken}
\usepackage{nkym}

% \analogtrue
% \colorfalse

\begin{document}

%% - - - - - - - - - - - - - - - - - %%
\vctitle{ブラックホールのエントロピー}% タイトル
\vcauthor{{中須}{かすみ}}% 名前
\vcbelong{理工研究科物理科学コース}% 所属
\vcgrade{M1}% 回生
%% - - - - - - - - - - - - - - - - - %%

\mokuji{2} % 目次出力
\vcmaketitle % タイトル出力

%% - - - - - - - - - 以下本文 - - - - - - - - - - - %%


%% - - - - - - - - - - - - - - - - - - - - - - - -
\section*{はじめに}
%% - - - - - - - - - - - - - - - - - - - - - - - -
こんにちわ.
ブラックホールの温度を計算してみます.

%% - - - - - - - - - - - - - - - - - - - - - - - -
\section{量子力学入門}
%% - - - - - - - - - - - - - - - - - - - - - - - -

%%
\subsection{系の時間発展の記述}

量子力学において,系の時間発展の記述の仕方には,次の2つの方法がある.
\begin{itemize}
  \item シュレーディンガー描像:物理量は時間発展せず,状態が時間発展する.
  \item ハイゼンベルク描像:状態は時間発展せず,物理量が時間発展する.
\end{itemize}

シュレーディンガー描像では,よく知られているように,
系の状態$\ket{\psi(t)}$はシュレーディンガー方程式
\begin{align}
  i\hbar \pdv{t} \ket{\psi(t)} = \hat H(t, \hat{\vc x}, \hat{\vc p}) \ket{\psi(t)}
\end{align}
に従って時間発展する.
$\hat H(t, \hat{\vc x}, \hat{\vc p})$は系のハミルトニアンである.
この方程式を差分化すると
\begin{align*}
  i\hbar \frac{\ket{\psi(t + \Delta t)} - \ket{\psi(t)}}{\Delta t}
  = \hat H(t, \hat{\vc x}, \hat{\vc p}) \ket{\psi(t)}
\end{align*}
となり,整理すると
\begin{align}
  \ket{\psi(t + \Delta t)}
  = \qty(1 - \frac{i}{\hbar} \Delta t \hat H(t, \hat{\vc x}, \hat{\vc p})) \ket{\psi(t)}
\end{align}
と書ける.
ここで,ハミルトニアン$\hat H$が時間に依存しないとすると,
時刻$t_0$から$t$までの状態$\ket{\psi(t_0)}$の時間発展は,
$t - t_0 = N \Delta t$として
\begin{align*}
  \ket{\psi(t)}
  &= \lim_{N\to\infty} \qty(1 - \frac{i}{\hbar} \frac{(t - t_0)}{N} \hat H(\hat{\vc x}, \hat{\vc p}))^N \ket{\psi(t_0)} \\
  &= e^{- \frac{i}{\hbar} (t - t_0) \hat H(\hat{\vc x}, \hat{\vc p})} \ket{\psi(t_0)}
\end{align*}
と書ける.

このとき,時刻$t_0$で$\hat O(t_0)$であった物理量の時刻$t$での期待値は
\begin{align}
  \expval*{\hat O(t)}
    \equiv \bra{\psi(t)} \hat O(t_0) \ket{\psi(t)}
    &= \bra{\psi(t_0)} e^{+ \frac{i}{\hbar} (t - t_0) \hat H} \hat O(t_0) e^{- \frac{i}{\hbar} (t - t_0) \hat H} \ket{\psi(t_0)}
\end{align}
と書けるが,ここで状態$\ket{\psi(t_0)}$は時刻$t$においても変化しておらず,
代わりに物理量$\hat O(t_0)$が$\hat O(t)$へと時間発展したとみなすと
\begin{align}
  \expval*{\hat O(t)}
  = \bra{\psi(t_0)} \hat O(t) \ket{\psi(t_0)}
\end{align}
と書ける.
ここで$\hat O(t)$は
\begin{align}
  \hat O(t) \equiv e^{+ \frac{i}{\hbar} (t - t_0) \hat H} \hat O(t_0) e^{- \frac{i}{\hbar} (t - t_0) \hat H}
\end{align}
である.
このように,系の状態が時間発展するのではなく,
代わりに物理量が時間発展するという記述の仕方をハイゼンベルク描像という.

%%
\subsection{混合状態}



%% - - - - - - - - - - - - - - - - - - - - - - - -
\section{統計力学入門}
%% - - - - - - - - - - - - - - - - - - - - - - - -

%%
\subsection{カノニカル分布}

以下では,温度$T$で熱平衡状態となっている系(熱浴と部分系)を考える.
すなわち,巨視的にみると熱浴と部分系の間でエネルギーの移動はないので,
部分系のハミルトニアンは定常(時間に依存しない)ということになる.

部分系の状態は,エネルギー固有状態$\ket{E_i}$が$\omega(E_i)$の割合で混合されているとすると,そこでの物理量$\hat A$の期待値は
\begin{align}
  \expval*{\hat A}
  &= \sum_i \omega(E_i) \bra{E_i} \hat A \ket{E_i} \\
  &= \sum_i \frac{e^{-\beta E_i}}{Z} \bra{E_i} \hat A \ket{E_i}
  = \frac1Z \sum_i \bra{E_i} \hat A e^{-\beta \hat H} \ket{E_i}
\end{align}
と書ける.

!!オメガの同定.

%%
\subsection{虚時間方向の周期}

逆温度$\beta$で熱平衡となっている系において,
任意の物理量の期待値は,虚時間方向に周期$\hbar\beta$を持つ.
すなわち
\begin{align}
  \expval*{\hat O(t)}_\beta
  = \expval*{\hat O(t - i\hbar\beta)}_\beta
  \label{eq:O-period}
\end{align}
が成り立つ.

証明は以下のようにして簡単に行うことができる.
逆温度$\beta$で熱平衡となっている系における,物理量$\hat O(t)$の期待値$\expval*{\hat O(t)}_\beta$は
\begin{align*}
  \expval*{\hat O(t)}_\beta
  &= \sum_n \frac{e^{-\beta E_n}}{Z} \bra{E_n} \hat O(t) \ket{E_n} \\
  &= \frac1Z \sum_n \bra{E_n} \hat O(t) e^{-\beta \hat H} \ket{E_n} \\
  &= \frac1Z \sum_n \bra{E_n} e^{-\beta \hat H} e^{+\frac{i}{\hbar}(-i\hbar\beta) \hat H} \hat O(t) e^{-\frac{i}{\hbar}(-i\hbar\beta) \hat H} \ket{E_n} \\
  &= \frac1Z \sum_n \bra{E_n} e^{-\beta \hat H} \hat O(t - i\hbar\beta) \ket{E_n} \\
    &= \expval*{\hat O(t - i\hbar\beta)}_\beta
\end{align*}
と書ける.
3行目から4行目への変形では,
$\hat O(t)$を解析接続することで虚時間方向への時間発展も許した;
\begin{align}
  e^{+\frac{i}{\hbar}(-i\hbar\beta) \hat H} \hat O(t) e^{-\frac{i}{\hbar}(-i\hbar\beta) \hat H}
  = \hat O(t - i\hbar\beta).
\end{align}
\QED

この事実は,後にブラックホールが存在する時空の虚時間方向が持つ周期から,その時空が持っている温度を同定する際に用いる.
すなわち,虚時間方向にある周期を持った時空は,外部から見て,ある有限温度を持った熱浴として観測されることになる.


%% - - - - - - - - - - - - - - - - - - - - - - - -
\section{相対論入門}
%% - - - - - - - - - - - - - - - - - - - - - - - -

\subsection{特殊相対論}

1次元の時間と3次元の(実)空間を合わせた4次元の抽象空間を時空と呼ぶ.
重力の存在しない平坦な時空はミンコフスキー時空と呼ばれ,
そこでの線素\footnote{限りなく近い2点間の距離}は直交座標で
\begin{align}
  \dd{s}^2
  = - (c\dd{t})^2 + \dd{x}^2 + \dd{y}^2 + \dd{z}^2
  \label{eq:ds-m-x}
\end{align}
と与えられる.
ここで$c$は光速度である.
% 空間部分は,ユークリッド幾何学におけるただの三平方の定理であるが,時間部分の前にはマイナス符号が付いている.

ミンコフスキー時空の線素は,空間部分を極座標で表せば
\begin{align}
  \dd{s}^2
  = - (c\dd{t})^2 + \dd{r}^2 + r^2 (\dd{\theta}^2 + \sin^2\theta \dd{\phi}^2)
  \label{eq:ds-m-r}
\end{align}
とも書ける.
\siki{ds-m-x},\siki{ds-m-r}は単なる変数変換で移り得るので,表式は異なるがどちらも同じ時空を表している.

線素\siki{ds-m-x}の各項の係数は計量と呼ばれ,ミンコフスキー時空の場合には
\begin{align}
  \eta_{\mu\nu} = \mqty(-1 & 0 & 0 & 0 \\ 0 & 1 & 0 & 0 \\ 0 & 0 & 1 & 0 \\ 0 & 0 & 0 & 1 \\)
\end{align}
という$4 \times 4$行列で与えられる.
これを用いて線素は
\begin{align*}
  \dd{s}^2
  &= - (c\dd{t})^2 + \dd{x}^2 + \dd{y}^2 + \dd{z}^2 \\
  &= \sum_{\mu=0}^3 \eta_{\mu\nu} \dd{x}^\mu \dd{x}^\nu \atag
\end{align*}
と書くことができる.
ここで$x^\mu$は
\begin{align}
  x^\mu \equiv \mqty(x^0 \\ x^1 \\ x^2 \\ x^3) \equiv \mqty(ct \\ x \\ y \\ z)
\end{align}
という4元ベクトルである.

また,便利なルールとしてアインシュタインの記法を用いることにする.
すなわち,上下に同じ添字が現れたときには自動で和をとることにする($\sum$記号を省略する).
例えば
\begin{align}
  a_\mu b^\mu
  \equiv \sum_{\mu=0}^3 a_\mu b^\mu
  = a_0 b^0 + a_1 b^1 + a_2 b^2 + a_3 b^3
\end{align}
といった具合である.
ここで,下付きの添字は,計量$\eta_{\mu\nu}$を用いて
\begin{align}
  a_\mu \equiv \eta_{\mu\nu} a^\nu
\end{align}
と定義する.

この記法を用いると線素は
\begin{align}
  \dd{s}^2
  = \eta_{\mu\nu} \dd{x}^\mu \dd{x}^\nu
  = \dd{x}_\mu \dd{x}^\mu
\end{align}
とシンプルな形で書くことができる.


\subsection{一般座標変換}

線素はどの慣性系にいる観測者から見ても(どの慣性座標系で測っても)同じ値をとることが知られている.



\subsection{リンドラー時空}
!!等加速度運動系として紹介するだけ.
ユークリッド化はここではしない.


\subsection{アインシュタイン方程式}



\subsection{シュワルツシルト解}

静的かつ球対称な計量$g_{\mu\nu}(x)$のanzats
\begin{align*}
  \dd{s}^2 = g_{00}(r) (c\dd{t})^2 + g_{11}(r) \dd{r}^2 + r^2 (\dd{\theta}^2 + \sin^2\theta \dd{\phi}^2)
\end{align*}
を仮定し,真空中のアインシュタイン方程式$R_{\mu\nu} = 0$を解くと,次のシュワルツシルト解が得られる;
\begin{align}
  \dd{s}^2 = - \qty(1 - \frac{r_h}{r}) (c\dd{t})^2 + \qty(1 - \frac{r_h}{r})^{-1} \dd{r}^2 + r^2 (\dd{\theta}^2 + \sin^2\theta \dd{\phi}^2).
  \label{eq:Schwarzschild}
\end{align}
ここで$r_h$はシュワルツシルト半径と呼ばれ,原点$r=0$に質量$M$の質量がある場合の万有引力の法則と比較することで
\begin{align}
  r_h = \frac{2GM}{c^2}
\end{align}
と同定される.
$G$は万有引力定数である.

このシュワルツシルト解は,質量$M$のブラックホールを表している.
シュワルツシルト半径$r_h$はブラックホールの半径とも見なされ,その球面は事象の地平面あるいは単にホライズンと呼ばれる.

シュワルツシルト解\siki{Schwarzschild}には2つの特異点$r = 0, r_h$が存在する.
ホライズン上$r = r_h$の方は,見かけの特異点であり,適切な座標変換を施せば除去可能である\footnote{例えば,}.
一方,原点$r = 0$の特異点は,座標変換で取り除くことはできない物理的な特異点である.
そのことは,座標変換で不変であるスカラー量を調べることで理解できる.
曲率テンソルから構成できるスカラー量のひとつとして,次のクレッチマン・スカラー
\begin{align}
  K \equiv R_{\mu\nu\rho\sigma} R^{\mu\nu\rho\sigma} = \frac{48 G^2 M^2}{c^4 r^6}
\end{align}
を見ると,確かに原点$r = 0$に特異点を持つことが見てとれる.


\subsection{ブラックホールの性質}

!!出られない話だけ?






%% - - - - - - - - - - - - - - - - - - - - - - - -
\section{ウンルー効果}
%% - - - - - - - - - - - - - - - - - - - - - - - -



%% - - - - - - - - - - - - - - - - - - - - - - - -
\section{ホーキング輻射}
%% - - - - - - - - - - - - - - - - - - - - - - - -




%% - - - - - - - - - - - - - - - - - - - - - - - -
%%  参考文献
%% - - - - - - - - - - - - - - - - - - - - - - - -
\begin{thebibliography}{99}
  \bibitem{nakasu1} 著者1・著者2,『本のタイトル』,出版社,出版年.
  \bibitem{nakasu2} ページの著者,『ページのタイトル』,最終アクセス日,\\(\url{https://vuccaken.github.io}).
\end{thebibliography}

\end{document} % - - - - - - - - - - - - - - - - - - - - -
%%
%% ファイトだよ!
%%
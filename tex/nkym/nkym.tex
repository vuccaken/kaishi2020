%%
%%   N  K  Y  M
%%    A  A  A  A
%%

%% typeset: latexmk (uplatex -> dvipdfmx)
\documentclass[uplatex,dvipdfmx]{vkaishi}

% \usepackage{v-hyperref}
\usepackage{vuccaken}
\usepackage{nkym}

% \analogtrue
% \colorfalse

\begin{document}

%% - - - - - - - - - - - - - - - - - %%
\vctitle{ブラックホールのエントロピー}% タイトル
\vcauthor{{中須}{かすみ}}% 名前
\vcbelong{理工研究科物理科学コース}% 所属
\vcgrade{M1}% 回生
%% - - - - - - - - - - - - - - - - - %%

\mokuji{2} % 目次出力
\vcmaketitle % タイトル出力

%% - - - - - - - - - 以下本文 - - - - - - - - - - - %%


%% - - - - - - - - - - - - - - - - - - - - - - - -
\section*{はじめに}
%% - - - - - - - - - - - - - - - - - - - - - - - -
こんにちわ.
ブラックホールの温度を計算してみます.

%% - - - - - - - - - - - - - - - - - - - - - - - -
\section{量子力学入門}
%% - - - - - - - - - - - - - - - - - - - - - - - -



%% - - - - - - - - - - - - - - - - - - - - - - - -
\section{統計力学入門}
%% - - - - - - - - - - - - - - - - - - - - - - - -


%% - - - - - - - - - - - - - - - - - - - - - - - -
\section{相対論入門}
%% - - - - - - - - - - - - - - - - - - - - - - - -

\subsection{特殊相対論}

\subsection{一般座標変換}

\subsection{リンドラー時空}
\subsubsection{Wick回転}

\subsection{アインシュタイン方程式}

\subsection{シュワルツシルト解}

\subsection{ブラックホールの性質}








%% - - - - - - - - - - - - - - - - - - - - - - - -
\section{一般相対論入門}
%% - - - - - - - - - - - - - - - - - - - - - - - -


%% - - - - - - - - - - - - - - - - - - - - - - - -
%%  参考文献
%% - - - - - - - - - - - - - - - - - - - - - - - -
\begin{thebibliography}{99}
  \bibitem{nakasu1} 著者1・著者2,『本のタイトル』,出版社,出版年.
  \bibitem{nakasu2} ページの著者,『ページのタイトル』,最終アクセス日,\\(\url{https://vuccaken.github.io}).
\end{thebibliography}

\end{document} % - - - - - - - - - - - - - - - - - - - - -
%%
%% ファイトだよ!
%%
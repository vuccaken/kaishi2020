%%
%%   TEMPLATE of kaishi
%%   コピペ用
%%

%% typeset: latexmk (uplatex -> dvipdfmx)
\documentclass[uplatex,dvipdfmx]{vkaishi}

% \usepackage{v-hyperref}
\usepackage{vuccaken}
\usepackage{kadono} % Your sty File

% \analogtrue
% \colorfalse

%% スタイルファイルの読み込み(\usepackage{})や自作マクロは,
%% プリアンブル(ここ)ではなく,自分のスタイルファイル(.sty)に書く.

\begin{document}

%% - - - - - - - - - - - - - - - - - %%
\vctitle{レーザーっておいしいの?}% タイトル
\vcauthor{\vname{門野}{広大}}% 名前
\vcbelong{理工学研究科基礎理工学専攻物理科学}% 所属
\vcgrade{2}% 回生
%% - - - - - - - - - - - - - - - - - %%

\mokuji{2} % 目次出力
\vcmaketitle % タイトル出力

%% - - - - - - - - - 以下本文 - - - - - - - - - - - %%


%% - - - - - - - - - - - - - - - - - - - - - - - -
\section*{はじめに}
%% - - - - - - - - - - - - - - - - - - - - - - - -
レーザーという言葉を聞いたことない人はほとんどいないと思います。しかし、このレーザーっていうものがどういうものなのかをきちんと理解している人は少ないのではないのでしょうか。レーザーというと


%% - - - - - - - - - - - - - - - - - - - - - - - -
\section{セクション}
%% - - - - - - - - - - - - - - - - - - - - - - - -
chapterは使わない\footnote{タイトル出力がchapterに相当します}.\par
section以下は自由に使ってよし.


%% - - - - - - - - - - - - - - - - - - - - - - - -
%%  参考文献
%% - - - - - - - - - - - - - - - - - - - - - - - -
%% \bibitem のlabelはユニークでなければいけないので,各自変更すること.
%% 著者名+出版年でラベリングするのがおすすめです.
\begin{thebibliography}{99}
  \bibitem{label-1} 著者1・著者2,『本のタイトル』,出版社,出版年.
  \bibitem{label-2} ページの著者,『ページのタイトル』,最終アクセス日,\\(\url{https://vuccaken.github.io}).
\end{thebibliography}

\end{document} % - - - - - - - - - - - - - - - - - - - - -
%%
%% ファイトだよ!
%%
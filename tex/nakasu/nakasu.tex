%%
%%   NAKASU
%%

%% typeset: latexmk (uplatex -> dvipdfmx)
\documentclass[uplatex,dvipdfmx]{vkaishi}

% \usepackage{v-hyperref}
\usepackage{vuccaken}
\usepackage{nakasu}

\begin{document}

%% - - - - - - - - - - - - - - - - - %%
\title{無敵級*ビリーバー}% タイトル
\author[中須]{かすみ}% 名前
\belong{普通科}{1}% 所属・回生
%% - - - - - - - - - - - - - - - - - %%

\mokuji{2} % 目次出力
\maketitle % タイトル出力

%% - - - - - - - - - 以下本文 - - - - - - - - - - - %%


%% - - - - - - - - - - - - - - - - - - - - - - - -
\section*{はじめに}
%% - - - - - - - - - - - - - - - - - - - - - - - -
やあ。


%% - - - - - - - - - - - - - - - - - - - - - - - -
\section{セクション1}
%% - - - - - - - - - - - - - - - - - - - - - - - -
ひん


%% - - - - - - - - - - - - - - - - - - - - - - - -
%%  参考文献
%% - - - - - - - - - - - - - - - - - - - - - - - -
%% \bibitem のlabelはユニークでなければいけないので,各自変更すること.
%% 著者名+出版年でラベリングするのがおすすめです.
\begin{thebibliography}{99}
  \bibitem{label-1} 著者1・著者2,『本のタイトル』,出版社,出版年.
  \bibitem{label-2} ページの著者,『ページのタイトル』,最終アクセス日,\\(\url{https://vuccaken.github.io}).
\end{thebibliography}

\end{document} % - - - - - - - - - - - - - - - - - - - - -
%%
%% ファイトだよ!
%%